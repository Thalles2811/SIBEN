\section{Fundamentos}
A intera\c{c}\~ao entre luz e mat\'eria inicia com a absor\c{c}\~ao de fot\~oes, promovendo as mol\'eculas a estados eletr\^onicos excitados. Os equil\'ibrios vibracionais se estabelecem rapidamente, resultando na emiss\~ao caracter\'istica quando a mol\'ecula retorna ao estado fundamental. A intensidade da fluoresc\^encia depende da efici\^encia do processo radiativo e das vias de desativa\c{c}\~ao n\~ao radiativas.

A absorb\^ancia de uma solu\c{c}\~ao segue a Lei de Lambert--Beer:
\begin{equation}
A = \epsilon c l,
\end{equation}
onde $\epsilon$ \'e o coeficiente de extin\c{c}\~ao molar, $c$ a concentra\c{c}\~ao e $l$ o caminho \'{o}ptico. A energia da radia\c{c}\~ao est\'a relacionada ao comprimento de onda por
\begin{equation}
E = h \nu = \frac{hc}{\lambda}.
\end{equation}

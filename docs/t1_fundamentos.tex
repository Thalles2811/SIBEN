% T\'opico 1: Fundamentos da Fluoresc\^encia
\documentclass[11pt]{article}
\usepackage[utf8]{inputenc}
\usepackage[T1]{fontenc}
\usepackage{amsmath}
\usepackage{graphicx}

\title{Fundamentos da Fluoresc\^encia}
\author{}
\date{\today}

\begin{document}
\maketitle

\begin{abstract}
Este texto apresenta os princ\'ipios b\'asicos da fluoresc\^encia, estruturando os conceitos de acordo com o roteiro descrito em \textit{Estrutura Geral}. A abordagem visa permitir compreens\~ao aut\^onoma do fen\^omeno e destacar suas aplica\c{c}\~oes.
\end{abstract}

\section{Introdu\c{c}\~ao do Fen\^omeno}
A fluoresc\^encia \'e identificada como a emiss\~ao de luz por uma mol\'ecula ap\'os absor\c{c}\~ao de radia\c{c}\~ao. De acordo com a lousa \textit{Estrutura Geral}, este tema deve ser situado no contexto da espectroscopia molecular, formulando o problema de compreender os processos de excita\c{c}\~ao e emiss\~ao. A revis\~ao da literatura evidencia contribui\c{c}\~oes cl\'assicas no estudo das transi\c{c}\~oes eletr\^onicas.

\section{Fundamentos F\'isico-Qu\'imicos}
O processo fundamental inicia-se com a absor\c{c}\~ao de fot\~oes, promovendo a mol\'ecula a um estado eletr\^onico excitado. A intera\c{c}\~ao entre os estados eletr\^onicos e vibracionais rege a subsequente emiss\~ao de radia\c{c}\~ao fluorescente. As transi\c{c}\~oes envolvem equil\'ibrios r\'apidos entre n\'iveis vibracionais, seguidos pela emiss\~ao a partir do estado eletr\^onico excitado.

\section{Equa\c{c}\~oes Essenciais}
A absor\c{c}\~ao de luz em solu\c{c}\~oes dilu\'idas segue a Lei de Lambert--Beer:
\begin{equation}
A = \epsilon c l,
\end{equation}
onde $A$ \'e a absorb\^ancia, $\epsilon$ o coeficiente de extin\c{c}\~ao molar, $c$ a concentra\c{c}\~ao e $l$ o caminho \'optico. A energia de uma radia\c{c}\~ao \'e dada por
\begin{equation}
E = h \nu = \frac{hc}{\lambda},
\end{equation}
relacionando a frequ\^encia $\nu$ ou o comprimento de onda $\lambda$ \`a energia transportada por cada fot\~ao.

\section{Justificativa Cient\'ifica e Aplica\c{c}\~oes}
O entendimento desses fundamentos permite avaliar efici\^encia de fluor\'oforos, essencial em diagn\'ostico e imageamento celular. A fluoresc\^encia tamb\'em sustenta o desenvolvimento de sensores e t\'ecnicas anal\'iticas de alta sensibilidade.

\section{Conclus\~ao}
Seguindo as etapas estruturais propostas, identificamos o tema, apresentamos fundamentos e equa\c{c}\~oes b\'asicas e discutimos aplica\c{c}\~oes. Estudos avan\c{c}ados podem explorar din\^amica e espectroscopia detalhadas.

\bibliographystyle{plain}
\bibliography{referencias}
\end{document}
